\documentclass[]{jsarticle}

\usepackage{graphicx}
\usepackage[dvipdfmx]{color}
\usepackage{comment}

\usepackage{here}

\begin{document}

\title{日本史 後期中間試験}
\author{}
\date{}

\maketitle

\renewcommand{\labelenumi}{(\arabic{enumi})}

\begin{enumerate}
	\item 廃藩置県
	\item 家禄
	\item 賞典禄
	\item 秩禄
	\item 武士のプライドから看板を出さなかったから
		\\
	\item 士族の商法
	\item 士農工商
	\item 学校を建てたりするのにお金を使いたかったから。
	\item 秩禄奉還の法
	\item 金禄公債証書
	\item 秩禄処分
	\item 廃刀令
	\item 不平士族
	\item 佐賀の乱
	\item 征韓論に敗れて下野した前参議・司法郷の江藤新平と約12000名
	\item 敬神党の乱
	\item 肥後藩士族の国学や神道を基本とした教育を重視する派閥の一部
	\item 太田黒伴雄を中心とした熊本県士族約190名
	\item 熊本鎮台
	\item 地方を守るために駐留している軍隊
	\item 熊本鎮台兵
	\item 秋月の乱
	\item 宮崎車之助ら旧秋月藩士約230名
	\item 熊本鎮台兵
	\item 萩の乱
	\item 前参議の前原誠一を率いる山口県士族数百名
	\item 県庁
	\item 広島鎮台兵
	\item 西南戦争
	\item 西郷隆盛と鹿児島の私学校生徒約3万人
	\item 政府軍約6万人
	\item 約8ヶ月間
	\item 加賀藩で参勤交代の費用の計算と海軍省で求められている仕事が似ていたから。
	\item 15\%
	\item 3万円
	\item 身分費用
	\item 海軍の制服
	\item 洋服がとりいれられてからまだそんなにたっていなかったから
	\item 土地をかって地主をやる
	\item 60円
	\item 官僚軍人になってほしかった
	\item 子供に字を書かせ、郵便で送らせて採点した。
	\item 島田一郎
	\item 石川県士族
	\item 子供に学問をさせて、海軍に入れてほしい
	\item 由緒、家柄
	\item 身分が下がったことや藩がなくなり評価されなくなったから
	\item 実力
		\\
	\item 昭和天皇実録
	\item 宮内庁
	\item 24年
	\item 昭和天皇の公式の伝記
	\item ある人は、「国民を思い、戦争中も平和を求めていた。」といい、ある人は「保身のために国民を犠牲にしていた。」と評価していた。
	\item 戦争責任
	\item 1941年12月8日
	\item 1945年8月15日
	\item じり貧論
	\item 臥薪嘗胆
	\item 目的達成のために努力・苦心を重ねること。
	\item 軍令部の最高責任者で、勅命を各部隊に伝達し、作戦を統率する階級
		\\
	\item えと・かんし
	\item 十干十二支
	\item 癸卯
	\item みずのとう・きぼう
	\item 甲辰
	\item きのえたつ・こうしん
	\item 丙犬
	\item ひえのいぬ・へいじゅつ
	\item 丁亥
	\item ひのとい・ていがい
	\item 60ある干支を一回りした年齢
	\item 人を呪い殺すために丑の刻に神社の裏で藁人形にトンカチで五寸釘を打ち付ける儀式
	\item 子午線は南北を結ぶ線という意味で、干支では北を「子」、南を「午」とするから。
	\item 甲子の年である1924年に作られたから。「甲子」は干支のはじめで縁起がいいとされているから。
		\\
	\item 大勝利はもちろん、勝つかどうかもおぼつかない
	\item GDP(国内総生産)
	\item 艦艇4.5倍、飛行機6倍、自動車450倍、アルミ6倍、鉄鋼10倍
	\item 鉄20倍、石油100倍、石炭10倍、電力6倍
	\item 日清戦争日露戦争に勝利(?)し、日露戦争については10倍の差があった。なので、戦争はやってみなくてはわからないという考えがあったから。加えて、じり貧なので、より貧しくなる前に開戦した。
	\item 「ただ今研究中のため、いずれ申し上げます」
	\item ソ連の参戦
	\item 登記戦争なので雪などに阻まれるので、ソ連は攻めてこない
	\item 陸軍の作戦を総括する役職
	\item 5ヶ月
	\item 陸軍大臣
	\item 支那事変
	\item 速戦即決
	\item 「支那の奥地が開けて広大でありますため、予定通り作戦が進みませんでした。
	\item 「支那の奥地が広大というなら、太平洋はもっと広いではないか。どのような確信があって5ヶ月というのか。」
	\item 統帥権
	\item 230万人
	\item 80万人
	\item 1945年8月
	\item 餓死
	\item 病気
	\item 自殺
	\item 70\%
	\item 制海、制空権を喪失し、物資の補給ができなかったから
	\item 軍備品の輸送や補給をすること
	\item 年寄りによる若者いじめがあったから
		\\
	\item 無条件降伏
	\item 一撃講和論
	\item 運命をかけて大勝負をして勝利し、少しでも有利な条件で講和に持ち込むこと
		\\
	\item 武装解除と戦争責任者問題
	\item 木戸幸一
	\item 木戸孝允
	\item 最高の側近
	\item 東京大空襲
	\item 皇族へ「無条件降伏と戦争責任者への処罰以外は、戦争終結の条件として考えられる。」
	\item ヒトラーの自決
	\item 外務大臣へ「早期終戦を希望する。」
	\item 御前会議で徹底抗戦と本土決戦が決定された
	\item 持たない
	\item 天皇は立憲君主であり、大日本帝国憲法で決められていたことは守る必要があったから。
	\item 輔弼
	\item 明治憲法の概念で、天皇の行為や決定に進言し、その結果に責任を負うこと。
	\\
	\item できない
	\item 会議前の質問
	\item 反対されるとわかっていたから。
	\item 懇談
	\item 政府・陸軍・海軍はそれぞれ責任があるが、ほかに対しては責任を持たなかったから。
	\item ポツダム宣言
	\item 宣言に名前のなかったソ連による仲介と具体的な期限の記載がないことによる返事の先延ばし
	\item ポツダム宣言に記されていた条件ではなく、選択の余地はなかった。かつ、ソ連はヤルタ秘密協定で対日参戦や南樺太、千島列島の譲渡、旅順、大連の自由港化を約していたから。また、返事は迅速にと記載されていたから。
	\item 広島への原爆投下
	\item ソ連の日本への宣戦布告
	\item 和平工作の完全失敗
	\item いいえ
	\item 広島への原爆投下
	\item 長崎への原爆投下
	\item ポツダム宣言の受諾
	\item 国家の体裁に対する考えが人によって違ったから。
	\item 将来発展する根基
	\item 鈴木貫太郎が御前会議で天皇の意見を聞くことを決定し、天皇が憲法に違反しない形で終戦という聖断をした。
	\\
	\item 真田穣一郎少将日記
	\item 神風特別攻撃隊
	\item 戦藻録
	\item 天皇の具体的な発言が記載されておらず、戦争に対する積極的な発言とみなされるものは消されているように見える。実録に書かれていないことは歴史的事実とみなされなくなる可能性がある。そして、それらは忘れ去られてしまう可能性がある。
	\item 史料批判を行い、事実関係を確認する
	\item 戦争の記憶の継承における到達点と欠落点
	\item 共有しよう!
\end{enumerate}

\end{document}
