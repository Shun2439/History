\documentclass[]{jsarticle}

\usepackage{array}
\usepackage{siunitx}
\usepackage{graphicx}
\usepackage[dvipdfmx]{color}
\usepackage{amsmath}

%コメント用のパッケージ
\usepackage{comment}

%画像を強制的に配置する
\usepackage{here}

\begin{document}

\title{日本史 後期期末試験}

\author{}

\date{}

\maketitle

\renewcommand{\labelenumi}{(\arabic{enumi})}

\begin{enumerate}
	\item 明治に入り、藩が県等に変わったことをなんというか?
	\item 華族と士族の身分は何になったか?\label{karoku}
	\item 王政復興の功労者にはなにが与えられたか?\label{shotenroku}
	\item \ref{karoku}と\ref{shotenroku}を合わせてなんというか?
	\item 明治に入り武士は店を始めたが、なぜすぐに廃業してしまったのか?\label{puraido}
	      \\
	\item \ref{puraido}などによって商売がうまくいっていなかったことをなんというか?
	\item 明治時代前の身分制度を表す言葉を答えよ。
	\item 政府はなぜ家禄をなくしたかったのか?
	\item 希望者のみに一時金を出し、秩禄を止めたことをなんというか?
	\item 政府は次に3~14年分の何を渡したか?\label{kinrokukousaishousho}
	\item \ref{kinrokukousaishousho}を渡したことをなんというか?\label{titsurokushobun}
	\item \ref{titsurokushobun}に加えて政府は何を定めたか?
	\item 上記の法によって何が生まれたか?\label{huheshizoku}
	\item \ref{huheshizoku}たちが起こした最初の反乱を答えよ。\label{saganoran}
	\item \ref{saganoran}はだれが中心となって起こったか?
	\item \ref{saganoran}の次に起きた反乱を答えよ。\label{keshintonoran}
	\item \label{keshintonoran}に含まれる単語の意味は何か?
	\item \ref{keshintonoran}は誰が中心に起こったか?
	\item \ref{keshintonoran}でどこが襲われたか?\label{kumamototindai}
	\item \ref{kumamototindai}などは何のことをいっているか?
	\item \ref{keshintonoran}はだれに鎮圧されたか?
	\item \ref{keshintonoran}の次に起きた士族による反乱は何か?\label{akidukinoran}
	\item \ref{akidukinoran}は誰が中心に起こったか?
	\item \ref{akidukinoran}はだれに鎮圧されたか?
	\item \ref{akidukinoran}の次に起きた士族による反乱は何か?\label{haginoran}
	\item \ref{haginoran}はだれが中心に起こったか?
	\item \ref{haginoran}はどこに向かっている途中に鎮められたか?
	\item \ref{haginoran}はだれに鎮圧されたか?
	\item \ref{haginoran}の次に起こった士族による反乱を答えよ。\label{senansenso}
	\item \ref{senansenso}はだれが中心となって起こったか?
	\item \ref{senansenso}で勝ったのはだれか?
	\item \ref{senansenso}はどのくらい行われたか?
	\item なぜ成之は海軍省に出仕できたのか?
	\item 官職に着けた士族の割合を答えよ
	\item 当時の1円は現在で何円か?
	\item 身分を保つための費用をなんというか?\label{mibunhiyo}
	\item 成之の場合、\ref{mibunhiyo}の大半を何に使っていたか?無文\label{seihuku}
	\item \ref{seihuku}はなぜ高価だったのか?
	\item 猪山家は余剰金を何に使おうとしたか?\label{hudosan}
	\item 1200円を\ref{hudosan}にかけたとすると年にいくら帰ってくるか?
	\item 成之は子供をどうしたかったか?\label{to_be_kanryogunjin}
	\item 成之は\ref{to_be_kanryogunjin}のために子供に何をさせたか?
	\item 大久保利通を暗殺したのはだれか?\label{itiro}
	\item \ref{itiro}はどこのなにか?
	\item \ref{itiro}が遺言書に残したことはなにか?
	\item 江戸時代に生きていくために重要だったことはなにか?\label{gomi}
	\item \ref{gomi}はなぜ、明治になり無効になったのか?
	\item \ref{gomi}ではなく、なにで評価されるようになったか?
	      \\
	\item 昭和天皇の生涯を記録した本をなんというか?\label{showatennojutsuroku}
	\item \ref{showatennojutsuroku}はどこが作ったか?
	\item \ref{showatennojutsuroku}の作成には何年かかったか?
	\item \ref{showatennojutsuroku}の作成のされかたからなんということができるか?
	      % \item 昭和天皇は5歳の時に何を飼っていたか?
	\item 昭和天皇への評価を答えよ。\label{hyoka}
	\item \ref{hyoka}に関して、当時昭和天皇のなにが考えられたか?
	\item 太平洋戦争が始まったのはいつか?
	\item 太平洋戦争が終戦したのはいつか?
	\item 太平洋戦争の開戦の際に唱えられていたことはなにか?
	\item 戦時下での日本の合言葉を答えよ\label{gashinshotan}
	\item \ref{gashinshotan}の意味を答えよ
	\item 軍令部総長とは何か?
\end{enumerate}

\begin{comment}
\begin{figure}[htbp]
	\begin{center}
		\includegraphics[width=100mm]{./src/RC_series_circuit_c.png}
		\caption{RC直列回路 ($V_C$の測定)}
		\label{fig:RC_series_circuit_c}
	\end{center}
\end{figure}
\begin{equation}
	\label{Relationship_between_impedance_Z}
	V = ZI [\si{\volt}], I = \frac{V}{Z} [\si{\ampere}], Z = \frac{V}{I}[\si{\ohm}]
\end{equation}
\begin{table}[h]
	\caption{計測および実験補助器具}
	\label{tab:fixtures}
	\centering
	\begin{tabular}{|c|c|c|c|}
		\hline
		器具名 & 製造元 & 計器番号 & 定格 \\
	\end{tabular}
\end{table}
\end{comment}

\end{document}
